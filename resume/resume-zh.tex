%%
%% Copyright (c) 2018-2019 Weitian LI <wt@liwt.net>
%% CC BY 4.0 License
%%
%% Created: 2018-04-11
%%

% Chinese version
\documentclass[zh]{resume}

% Adjust icon size (default: same size as the text)
\iconsize{\Large}

% File information shown at the footer of the last page
\fileinfo{%
  \faCopyright{} 2019, 黄珂邈 \hspace{0.5em}
  \githublink{kemo-huang}{kemo-cv} \hspace{0.5em}
  \faEdit{} \today
}

\name{珂邈}{黄}

\keywords{自动驾驶, 计算机视觉, 软件工程, 传感器融合}

% \tagline{\icon{\faBinoculars}} <position-to-look-for>}
% \tagline{<current-position>}

% \photo{<height>}{<filename>}

\profile{
  \mobile{189-2935-7397}
  \email{11610728@mail.sustech.edu.cn}
  \birthday{1998年11月} \\
  \home{广东深圳}
  \github{Kemo-Huang}
  % Custom information:
  % \icontext{<icon>}{<text>}
  % \iconlink{<icon>}{<link>}{<text>}
}

\begin{document}
\makeheader

%======================================================================
\sectionTitle{教育背景}{\faGraduationCap}
%======================================================================
\begin{educations}
  \education%
    {2016年9月}%
    {2020年6月}%
    {南方科技大学}%
    {深圳}%
    {计算机科学与工程系}%
    {计算机科学与技术}%
    {工程学士}%
	{\begin{itemize}%
		\item GPA: 3.53/4.0, 三次获得奖学金
		\item 计算机系创新实验成果展三等奖
		\item 树礼书院学生会宣传部部长,2018最佳部门
		\item 翰墨社创始人、社长
	\end{itemize}
	专业课程:数据结构与算法分析、概率论与数理统计、计算机网络、计算机组成原理、嵌入式系统与微机原理、数据库原理、面向对象分析与设计、软件工程、软件测试、离散数学、人工智能、智能机器人、计算机视觉、机器学习
	}%
\end{educations}

%======================================================================
\sectionTitle{研究领域}{\faStar}
%======================================================================
\begin{itemize}
  \item \textbf{自动驾驶 \& 机器人}\\
  	感知系统, 传感器融合, 运动估计, 定位与建图
  \item \textbf{机器学习 \& 计算机视觉}\\
  	深度学习, 贝叶斯与统计, 图像与点云处理, 目标检测和跟踪
\end{itemize}

%======================================================================
\sectionTitle{学术经历}{\faAtom}
%======================================================================

\textit{南科大智能感知与无人系统实验室}

\begin{experiences}
  \experience%
    {目前}%
    {三维车辆跟踪}%
    [同时利用目标的运动和外貌信息实现自动驾驶中的多目标跟踪。使用卡尔曼滤波器对车辆进行三维运动建模,利用相机与激光雷达特征融合提高视觉算法准确度。设计有效的跟踪算法流水线,减少计算冗余并增加系统对目标丢失和遮挡的健壮性。]

\separator{0.5ex}
  \experience%
    {2019.10}%
    {三维车辆检测}%
    [基于Linux和ROS,对自动驾驶软件Autoware集成前沿的三维目标检测算法。利用阿尔法巴公司采集的巴士数据对软件和算法进行测试。] 

\separator{0.5ex}
  \experience%
    {2019.5}%
    {视觉传感器的标定}%
    [对单目相机和激光雷达进行外参标定。使用边缘提取算法分别建立2D和3D的对应点,通过RANSAC优化后的PnP方法求得外参。]

\separator{0.5ex}
  \experience%
    {2018.10}%
    {激光雷达点云超分辨率重建}%
    [优化了深度图像处理的流水线,分阶段使用核矩阵对投影后的点云图像进行扩张、非重要区域舍弃和异常信息过滤。]   
    

\end{experiences}

%%======================================================================
\sectionTitle{编程技能}{\faCogs}
%======================================================================
\begin{competences}[12em]
  \comptence{操作系统}{
    Linux, Robot Operating System (ROS)
  }
  \comptence{软件库}{%
    OpenCV, Pytorch, Scikit-learn, Pandas, MATLAB, Point Cloud Library (PCL)
  }
  \comptence{应用开发}{%
    安卓, SpringBoot, Flask, 微信小程序, Unity
  }
\end{competences}


\end{document}
