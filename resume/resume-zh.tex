%%
%% Copyright (c) 2018-2019 Weitian LI <wt@liwt.net>
%% CC BY 4.0 License
%%
%% Created: 2018-04-11
%%

% Chinese version
\documentclass[zh]{resume}

% Adjust icon size (default: same size as the text)
\iconsize{\Large}

% File information shown at the footer of the last page
\fileinfo{%
  \faCopyright{} 2019, 黄珂邈 \hspace{0.5em}
  \githublink{kemo-huang}{kemo-cv} \hspace{0.5em}
  \faEdit{} \today
}

\name{珂邈}{黄}

% \tagline{\icon{\faBinoculars}} <position-to-look-for>}
% \tagline{<current-position>}

% \photo{<height>}{<filename>}

\profile{
  \mobile{189-2935-7397}
  \email{11610728@mail.sustech.edu.cn}
  \birthday{1998年11月} \\
  \home{广东深圳}
  \github{Kemo-Huang}
  \webpage{kemiao.webiste}
  % Custom information:
  % \icontext{<icon>}{<text>}
  % \iconlink{<icon>}{<link>}{<text>}
}

\begin{document}
\makeheader

%======================================================================
\sectionTitle{教育背景}{\faGraduationCap}
%======================================================================
\begin{educations}
  \education%
    {2020年9月}%
    {2023年7月}%
    {南方科技大学}%
    {深圳}%
    {计算机科学与工程系}%
    {电子科学与技术}%
    {硕士}%
	{\begin{itemize}%
		\item 课题组导师:郝祁(工学院人工智能无人驾驶创新平台)
	\end{itemize}
	专业课程:数据结构与算法分析、概率论与数理统计、计算机网络、计算机组成原理、嵌入式系统与微机原理、数据库原理、面向对象分析与设计、软件工程、软件测试、离散数学、人工智能、智能机器人、计算机视觉、机器学习
	}%
  \education%
    {2016年9月}%
    {2020年7月}%
    {南方科技大学}%
    {深圳}%
    {计算机科学与工程系}%
    {计算机科学与技术}%
    {学士}%
	{\begin{itemize}%
		\item GPA: 3.53/4.0, 三次获得奖学金
		\item 计算机系创新实验成果展三等奖
		%\item 树礼书院学生会宣传部部长,2018最佳部门
		%\item 翰墨社创始人、社长
	\end{itemize}
	专业课程:数据结构与算法分析、概率论与数理统计、计算机网络、计算机组成原理、嵌入式系统与微机原理、数据库原理、面向对象分析与设计、软件工程、软件测试、离散数学、人工智能、智能机器人、计算机视觉、机器学习
	}%
\end{educations}

%======================================================================
\sectionTitle{研究方向}{\faStar}
%======================================================================
\begin{itemize}
  \item \textbf{自动驾驶感知}\\
  	3D目标检测、跟踪,相机与激光雷达感知融合
\end{itemize}

%======================================================================
\sectionTitle{学术经历}{\faAtom}
%======================================================================

\begin{experiences}

  \experience%
    {目前的方向)}%
    {3D目标检测}%
    [使用Autoware(ROS)和预训练的PointPillars网络对阿尔法巴士采集的点云数据进行在线测试。] 
    
\separator{0.5ex}
  \experience%
    {2021.3(硕士)}%
    {3D目标检测与跟踪}%
    [为了改进目标跟踪算法性能,联合了多模态检测(EPNet)和ReID两部分的网络,并使用MIP的方式对网络输出的分类置信度、匹配置信度、进入和遮挡置信度,以及卡尔曼滤波器的预测结果进行综合的数据联合,在KITTI数据集上产生了更鲁棒的跟踪效果。以第一作者在IROS2021发表论文``Joint Multi-Object Detection and Tracking with Camera and LiDAR Fusion for Autonomous Driving'']

\separator{0.5ex}
  \experience%
    {2019.10(本科)}%
    {3D目标检测}%
    [使用Autoware(ROS)和预训练的PointPillars网络对阿尔法巴士采集的点云数据进行在线测试。] 

\separator{0.5ex}
  \experience%
    {2019.5(本科)}%
    {相机和激光雷达外参标定}%
    [为了融合多模态数据,使用边缘提取算法分别建立2D和3D的对应点,通过PnP方法求外参,在几何上实现传感器校准。]

\separator{0.5ex}
  \experience%
    {2018.10(本科)}%
    {点云超分辨率重建}%
    [为了解决点云数据稀疏问题,分阶段使用核矩阵对投影后的点云图像进行扩张、非重要区域舍弃和异常信息过滤,生成稠密的伪点云。]   
\end{experiences}

%%======================================================================
\sectionTitle{其他介绍}{\faCogs}
%======================================================================
\begin{competences}[12em]
  \comptence{英语}{
    CET-6, TOFEL (93)
  }
  \comptence{开发能力}{
    熟悉Python, PyTorch和Linux, 对C/C++, ROS, sklearn, OpenCV, PCL以及Java软件开发有一定程度的工程经验。
  }
\end{competences}


\end{document}
