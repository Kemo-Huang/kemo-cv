%%
%% Copyright (c) 2018-2019 Weitian LI <wt@liwt.net>
%% CC BY 4.0 License
%%
%% Created: 2018-04-11
%%

% Chinese version
\documentclass[zh]{resume}

% Adjust icon size (default: same size as the text)
\iconsize{\Large}

% File information shown at the footer of the last page
\fileinfo{%
  \faCopyright{} 2022, 黄珂邈 \hspace{0.5em}
  \githublink{Kemo-Huang}{kemo-cv} \hspace{0.5em}
  \faEdit{} \today
}

\name{珂邈}{黄}

% \tagline{\icon{\faBinoculars}} <position-to-look-for>}
% \tagline{<current-position>}

% \photo{<height>}{<filename>}

\profile{
  \mobile{189-2935-7397}
  \email{12032943@mail.sustech.edu.cn}
  \birthday{1998年11月} \\
  \home{广东深圳}
  \github{Kemo-Huang}
  % \webpage{kemiao.webiste}
  % Custom information:
  % \icontext{<icon>}{<text>}
  % \iconlink{<icon>}{<link>}{<text>}
}

\begin{document}
\makeheader

%======================================================================
\sectionTitle{教育背景}{\faGraduationCap}
%======================================================================
\begin{educations}
  \education%
    {2020年9月}%
    {2023年7月}%
    {南方科技大学}%
    {深圳}%
    {计算机科学与工程系}%
    {电子科学与技术}%
    {硕士}%
	{\begin{itemize}%
		\item 课题组导师:郝祁(工学院人工智能无人驾驶创新平台)
	\end{itemize}
	}%
  \education%
    {2016年9月}%
    {2020年7月}%
    {南方科技大学}%
    {深圳}%
    {计算机科学与工程系}%
    {计算机科学与技术}%
    {学士}%
	{\begin{itemize}%
		\item GPA: 3.53/4.0,四次获得奖学金
		\item 学生会部长,社团创始人
		%\item 树礼书院学生会宣传部部长,2018最佳部门
		%\item 翰墨社创始人、社长
	\end{itemize}
	专业课程:数据结构与算法分析、概率论与数理统计、计算机网络、计算机组成原理、嵌入式系统与微机原理、数据库原理、面向对象分析与设计、软件工程、软件测试、离散数学、人工智能、智能机器人、计算机视觉、机器学习、优化算法
	}%
\end{educations}

%======================================================================
\sectionTitle{研究方向}{\faStar}
%======================================================================
\begin{itemize}
  \item \textbf{自动驾驶感知}\\
  	3D目标检测、跟踪,相机与激光雷达感知融合
\end{itemize}

%======================================================================
\sectionTitle{学术经历}{\faAtom}
%======================================================================

\begin{experiences}

  \experience%
    {目前方向}%
    {3D目标检测与跟踪(新)}%
    [为了更好地解决目标跟踪时因遮挡导致的漏检和目标间的错误匹配等问题,(1) 拟在检测网络中加入RNN模型,(2) 进行ReID和速度预测的辅助训练,(3) 传感器特征融合时添加可学习的校准参数,(4) 加入场景流补充被遮挡点云,来最大化地增强数据完整度和网络模型的全局attention。] 
    
\separator{0.5ex}
  \experience%
    {2021.3(硕)}%
    {3D目标检测与跟踪}%
    [为了改进目标跟踪算法性能,在ROI层面联合多模态检测网络(EPNet)和correlation的网络,并使用MIP的方式对网络输出的分类置信度、匹配置信度、进入和遮挡置信度,以及卡尔曼滤波器的预测结果进行综合地数据联合,在KITTI数据集上实现了更鲁棒的跟踪效果和更快的速度,准确度在当时的leaderboard上排名第四,并以第一作者发表IROS 2021论文``Joint Multi-Object Detection and Tracking with Camera and LiDAR Fusion for Autonomous Driving'']

\separator{0.5ex}
  \experience%
    {2019.10(本)}%
    {3D目标检测}%
    [使用Autoware(ROS)和预训练的PointPillars网络对巴士采集的点云数据进行在线测试。] 

\separator{0.5ex}
  \experience%
    {2019.5(本)}%
    {相机、激光雷达外参标定}%
    [为了融合多模态数据,使用边缘提取算法分别建立2D和3D的对应点,通过PnP方法求外参,在几何上实现传感器校准。]

\separator{0.5ex}
  \experience%
    {2018.10(本)}%
    {点云超分辨率重建}%
    [为了解决点云稀疏问题,分阶段对投影后的点云图像进行扩张、非重要区域舍弃和异常信息过滤,生成稠密的伪点云。]   
\end{experiences}

%%======================================================================
\sectionTitle{其他介绍}{\faCogs}
%======================================================================
\begin{competences}[12em]
  \comptence{英语水平}{
    CET-6, TOFEL (93分)
  }
  \comptence{开发能力}{
    掌握Python、PyTorch和Linux, 熟悉OpenPCDet等开源项目,对C/C++、ROS、sklearn、OpenCV、PCL以及Java软件开发有一定程度的使用经验。
  }
\end{competences}


\end{document}
