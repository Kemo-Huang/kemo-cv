% English version
\documentclass{resume}

% Adjust icon size (default: same size as the text)
\iconsize{\Large}

% File information shown at the footer of the last page
\fileinfo{%
  \faCopyright{} 2019, Kemiao HUANG \hspace{0.5em}
  \githublink{kemo-huang}{kemo-cv} \hspace{0.5em}
  \faEdit{} \today
}

\name{Kemiao}{Huang}

\keywords{Autonomous Driving, Computer Vision, Software Engineering, Sensor Fusion}

% \tagline{\icon{\faBinoculars}} <position-to-look-for>}
% \tagline{<current-position>}

% \photo{<height>}{<filename>}

\profile{
  \mobile{(+86)189-2935-7397}
  \email{11610728@mail.sustech.edu.cn}
  \birthday{Nov. 1998} \\
  \home{Shenzhen, Guangdong, China}
  \github{Kemo-Huang}
  % Custom information:
  % \icontext{<icon>}{<text>}
  % \iconlink{<icon>}{<link>}{<text>}
}

\begin{document}
\makeheader

%======================================================================
\sectionTitle{EDUCATION}{\faGraduationCap}
%======================================================================
\begin{educations}
  \education%
    {Sep. 2016}%
    {Jun. 2020}%
    {Southern University of Science and Technology}%
    {Shenzhen, China}%
    {Department of Computer Science and Engineering}%
    {Computer Science}%
    {Bachelor of Engineering}%
	{\begin{itemize}%
		\item Cumulative GPA: 3.53/4.0, three scholarships
		\item Third prize winner for the exhibition of Innovative Experiment at CSE.
		\item Head of publicity department at student union with the 2018 best department award
		\item Founder of college calligraphy and painting club
	\end{itemize}}%
\end{educations}
	
%======================================================================
\sectionTitle{ACADEMIC INTEREST}{\faStar}
%======================================================================
\begin{itemize}
  \item \textbf{Autonomous Driving \& Robotics}\\
  	Perception System, Sensor Fusion, Behavior Prediction, Simultaneous Localization and Mapping (SLAM).
  \item \textbf{Machine Learning \& Computer Vision}\\
  	Deep Learning, Bayesian and Statistics, Image and Point Cloud Processing, Object Detection and Tracking.
\end{itemize}

%======================================================================
\sectionTitle{ACADEMIC EXPERIENCE}{\faAtom}
%======================================================================

Research Lab: \textit{Intelligent Sensing and Unmanned Systems}, SUSTech

\begin{experiences}
  \experience%
    {Present}%
    {3D Vehicle Tracking}%
    [Improved the robustness of the algorithm by fusing camera and LiDAR data with attention mechanism for tracking 3D vehicles. Devised an efficient tracking-by-detection pipeline for the image-and-point appearance model and the motion model.]

\separator{0.5ex}

  \experience%
    {Oct. 2019}%
    {3D Vehicle Detection}%
    [Built an online 3D vehicle detection system for monocular cameras and a single LiDAR. Integrated the detection algorithm with the "Autoware" software for autonomous driving system based on robot operating system (ROS) on Linux and tested with bus data from Shenzhen Haylion Tech.] 

\separator{0.5ex}

  \experience%
    {May 2019}%
    {Sensor Calibration}%
    [Experimented extrinsic calibration between monocular cameras and LiDARs. The 2D-3D point correspondences are built by corner detectors respectively and the extrinsic parameters are solved efficently by perspective-n-point (PnP) methods with RANSAC optimization.]

\separator{0.5ex}

  \experience%
    {Oct. 2018}%
    {LiDAR Point Cloud Upsampling}%
    [Assessed and optimized an image processing pipeline for depth image super-resolution. The depth from point clouds are completed by kernel dilation and exception filtering with fast speed and relatively low loss upon KITTI benchmark.]   
    

\end{experiences}

%%======================================================================
\sectionTitle{PROGRAMMING SKILLS}{\faCogs}
%======================================================================
\begin{competences}[12em]
  \comptence{Operating Systems}{
    Linux, Robot Operating System (ROS)
  }
  \comptence{Software Libraries}{%
    OpenCV, Pytorch, Scikit-learn, Pandas, MATLAB, Point Cloud Library (PCL)
  }
  \comptence{App Development}{%
    Android, SpringBoot, Flask, WeChat mini-app, Unity
  }
\end{competences}


\end{document}
