%%
%% Copyright (c) 2018-2019 Weitian LI <wt@liwt.net>
%% CC BY 4.0 License
%%
%% Résumé
%% ------
%% A short document (1-2 pages) to sum up the job-related accomplishments
%% and experience.
%%
%% Checklist
%% ---------
%% * Contact Information
%% * Work History / Experience
%% * Education
%% * Skills
%% * Summary & Objective (optional)
%% * Hobbies & Interests (optional)
%%
%% Credits
%% -------
%% * CV vs. Resume: What is the Difference? When to Use Which?
%%   https://uptowork.com/blog/cv-vs-resume-difference
%% * How to Make a Resume: A Step-by-Step Guide (+30 Examples)
%%   https://uptowork.com/blog/how-to-make-a-resume
%% * Entry-Level Resume: Sample and Complete Guide (+20 Examples)
%%   https://uptowork.com/blog/entry-level-resume-example
%%
%% Created: 2018-04-14
%%

% English version
\documentclass{resume}

% Adjust icon size (default: same size as the text)
\iconsize{\Large}

% File information shown at the footer of the last page
\fileinfo{%
  \faCopyright{} 2019, Kemiao HUANG \hspace{0.5em}
  \githublink{kemo-huang}{kemo-cv} \hspace{0.5em}
  \faEdit{} \today
}

\name{Kemiao}{Huang}

\keywords{Autonomous Driving, Computer Vision, Software Engineering, Sensor Fusion}

% \tagline{\icon{\faBinoculars}} <position-to-look-for>}
% \tagline{<current-position>}

% \photo{<height>}{<filename>}

\profile{
  \mobile{(+86)189-2935-7397}
  \email{11610728@mail.sustech.edu.cn}
  \birthday{Nov. 1998} \\
  \home{Shenzhen, Guangdong, China}
  \github{Kemo-Huang}
  % Custom information:
  % \icontext{<icon>}{<text>}
  % \iconlink{<icon>}{<link>}{<text>}
}

\begin{document}
\makeheader

%======================================================================
\sectionTitle{EDUCATION}{\faGraduationCap}
%======================================================================
\begin{educations}
  \education%
    {Sep. 2016}%
    {Present}%
    {Southern University of Science and Technology}%
    {Shenzhen, China}%
    {Department of Computer Science and Engineering}%
    {Computer Science}%
    {Bachelor of Engineering}%
	{\begin{itemize}%
		\item Cumulative GPA: 3.53/4.0, three scholarships
		\item Third prize winner for the exhibition of Innovative Experiment at CSE.
		\item Head of publicity department at student union with the 2018 best department award
		\item Founder of college calligraphy and painting club
	\end{itemize}}%
\end{educations}
	
%======================================================================
\sectionTitle{RESEARCH OF INTEREST}{\faStar}
%======================================================================
\begin{itemize}
  \item \textbf{Autonomous Driving \& Robotics}\\
  	Perception System, Vehicle Tracking, Sensor Fusion, Behavior Prediction, Simultaneous Localization and Mapping (SLAM), Mobile Robots.
  \item \textbf{Machine Learning \& Computer Vision}\\
  	Neural Networks, Bayesian Machine Learning, Image and Point Cloud Processing, Multiple Object Detection and Tracking.   	
\end{itemize}

%======================================================================
\sectionTitle{ACADEMIC EXPERIENCE}{\faAtom}
%======================================================================

Research Lab: \textit{Intelligent Sensing and Unmanned Systems}, SUSTech

\begin{experiences}
  \experience%
    {Present}%
    {3D Vehicle Tracking}%
    [Improved the robustness of a state-of-the-art tracking algorithm by fully fused camera and LiDAR data. Devised a tracking-by-detection pipeline with effective image-point feature extraction and appearance-motion modeling by long-short term memory (LSTM).]

\separator{0.5ex}

  \experience%
    {Oct. 2019}%
    {3D Vehicle Detection}%
    [Built an online 3D vehicle detection system for monocular cameras and LiDAR. Integrated the detection algorithm with the "Autoware" software for autonomous driving system based on robotic operating system (ROS). It's an internship project of Shenzhen Haylion Tech.] 

\separator{0.5ex}

  \experience%
    {May 2019}%
    {Sensor Calibration}%
    [Experimented extrinsic calibration between monocular cameras and LiDARs. The 2D-3D point correspondences are trained with neural networks and the extrinsic parameters are solved efficently by perspective-n-point (PnP) methods with RANSAC optimization.]

\separator{0.5ex}

  \experience%
    {Oct. 2018}%
    {LiDAR Point Cloud Upsampling}%
    [Assessed and optimized an image processing pipeline for depth image super-resolution. The depth from point clouds are completed by filter dilation and Markov random fields with fast speed and relatively low loss against ground truth of Kitti.]   
    

\end{experiences}

%%======================================================================
\sectionTitle{SKILLS}{\faCogs}
%======================================================================
\begin{competences}[12em]
  \comptence{Operating Systems}{
    Linux, Robotic Operating System (ROS)
  }
  \comptence{Programming Libraries}{%
    OpenCV, Pytorch, Scikit-learn, Pandas, MATLAB, Point Cloud Library (PCL)
  }
  \comptence{Web Development}{%
    Android, SpringBoot, Flask, WeChat mini-app, Unity
  }
\end{competences}


\end{document}
