% English version
\documentclass{resume}

% Adjust icon size (default: same size as the text)
\iconsize{\Large}

% File information shown at the footer of the last page
\fileinfo{%
  \faCopyright{} 2021, Kemiao HUANG \hspace{0.5em}
  \githublink{kemo-huang}{kemo-cv} \hspace{0.5em}
  \faEdit{} \today
}

\name{Kemiao}{Huang}

\keywords{Autonomous Driving, Computer Vision, Software Engineering, Sensor Fusion}

% \tagline{\icon{\faBinoculars}} <position-to-look-for>}
% \tagline{<current-position>}

% \photo{<height>}{<filename>}

\profile{
  \mobile{(+86)189-2935-7397}
  \email{12032943@mail.sustech.edu.cn}
  \birthday{Nov. 1998} \\
  \home{Shenzhen, Guangdong, China}
  \github{Kemo-Huang}
  % \webpage{kemiao.website}
  % Custom information:
  % \icontext{<icon>}{<text>}
  % \iconlink{<icon>}{<link>}{<text>}
}

\begin{document}
\makeheader

%======================================================================
\sectionTitle{EDUCATION}{\faGraduationCap}
%======================================================================
\begin{educations}
  \education%
    {Sep. 2016}%
    {Jun. 2020}%
    {Southern University of Science and Technology}%
    {Shenzhen, China}%
    {Department of Computer Science and Engineering}%
    {Computer Science}%
    {Bachelor of Engineering}%
	{\begin{itemize}%
		\item Cumulative GPA: 3.53/4.0, three scholarships.
	\end{itemize}}%
 \education%
    {Sep. 2020}%
    {Present}%
    {Southern University of Science and Technology}%
    {Shenzhen, China}%
    {Department of Computer Science and Engineering}%
    {Computer Science}%
    {Master of Engineering}%
	{\begin{itemize}%
		\item Published an IROS 2021 paper: ``Joint Multi-Object Tracking with Camera-LiDAR Fusion for Autonomous Driving''.
	\end{itemize}}%
	
\end{educations}
	
%======================================================================
\sectionTitle{RESEARCH INTEREST}{\faStar}
%======================================================================
\begin{itemize}
  \item \textbf{Autonomous Driving \& Computer Vision}\\
  	3D Object Detection and Tracking, Camera-LiDAR Fusion, Prediction.
\end{itemize}

%======================================================================
\sectionTitle{ACADEMIC EXPERIENCE}{\faAtom}
%======================================================================

Research Lab: \textit{Intelligent Sensing and Unmanned Systems}, SUSTech

\begin{experiences}
  \experience%
	{2021}%
    {Joint 3D Object Detection and Tracking}%
    [This work develops an joint network to perform object detection and correlation simultaneously with fused camera-LiDAR input. Detection confidences, affinities and start-end confidences are comprehensively used in data association. The experiment shows competitive results on the KITTI benchmark.]  
  
\separator{0.5ex}

  \experience%
    {2020}%
    {3D Object Tracking}%
    [Following the tracking-by-detection paradigm, this work studies the effectiveness of different camera-LiDAR fusion methods and motion prediction methods in multi-object tracking.]

\separator{0.5ex}

  \experience%
    {2019}%
    {3D Object Detection}%
    [This work builds an online 3D vehicle detection system with camera-LiDAR input based on ``Autoware'' and the robotic operating system. The system was tested with the real-word data.] 

\separator{0.5ex}

  \experience%
    {2019}%
    {Camera-LiDAR Calibration}%
    [This work studies the extrinsic calibration between cameras and LiDARs. The 2D-3D point correspondences are built by the corner detectors respectively. The parameters are finally solved by perspective-n-point (PnP) methods with RANSAC optimization.]

\separator{0.5ex}

  \experience%
    {2018}%
    {LiDAR Depth Image Completion}%
    [This work evaluates an image processing pipeline for depth image super-resolution. The sparse depth points are completed by the kernel dilation and the exception filtering.]   
    

\end{experiences}

%%======================================================================
\sectionTitle{PROGRAMMING SKILLS}{\faCogs}
%======================================================================
\begin{competences}[12em]
  \comptence{Programming Libraries}{%
    PyTorch, OpenCV, Scikit Learn, Point Cloud Library (PCL)
  }
  \comptence{Operating Systems}{
    Ubuntu, Robotic Operating System (ROS)
  }
  \comptence{Simulation Software}{%
    Carla, Autoware, Unity
  }
  \comptence{Web Frameworks}{%
    Android, SpringBoot, Flask, WeChat Mini-App
  }
\end{competences}


\end{document}
